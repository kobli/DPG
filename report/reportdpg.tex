\documentclass[report,11pt]{elsarticle}
%\documentclass{article}[11pt,a4]
%\documentclass[11pt]{article}

\usepackage[lined,linesnumbered]{algorithm2e}
\usepackage{a4wide}
\usepackage{ae}
\usepackage[czech,english]{babel}
\usepackage[utf8]{inputenc}
\usepackage{graphicx}
\usepackage{url}
\usepackage{pdfpages}

% fix for: czech babel breaks cline and cmidrule
\usepackage{regexpatch}
\makeatletter
% Change the `-` delimiter to an active character
\xpatchparametertext\@@@cmidrule{-}{\cA-}{}{}
\xpatchparametertext\@cline{-}{\cA-}{}{}
\makeatother

\paperwidth=210 true mm
\paperheight=297 true mm
\pdfpagewidth=210 true mm
\pdfpageheight=297 true mm

% Pro cestinu zakomentuj nasledujici radek
\selectlanguage{english}
% Pro anglictinu zakomentuj nasledujici radek
%\selectlanguage{czech}

% Znakem procenta zacina komentovaný řádek

%%%% Pokud chcete použít k formátování pseudokódu balík algorithm2e,
%%% odkomentujte jeden ze dvou následujících řádků
%\usepackage[lined,linesnumbered]{algorithm2e}
%\usepackage{algorithm2e}
% Tento balík slouží pro vkládání obrázků ve formátu EPS (encapsulated postscript)
%\usepackage{ctable}
%\DeclareGraphicsExtensions{.pdf}

% Remove the footer "Preprint submitted .."
\makeatletter
\def\ps@pprintTitle{%
	\let\@oddhead\@empty
	\let\@evenhead\@empty
	\def\@oddfoot{}%
	\let\@evenfoot\@oddfoot}

\begin{document}

\begin{frontmatter}

\title{Term project number 17\\ Hierarchical View-Frustum Culling for Z-buffer Rendering}

\author{Aleš Koblížek\footnote{B4M39DPG -- Aleš Koblížek, summer 2018/19}\\
Department of Computer Graphics and Interaction,\\ Faculty of Electrical Engineering, CTU in Prague
}

\date{}


%%%%%%%%%%%%%%%%%%%%%%%%%%%%%%%%%%%%%%%%%%%%%%%%%%%%%%%%%%%%%%%%%%%%%%%%%%%%%%
%%
%%  Abstract
%%
%%%%%%%%%%%%%%%%%%%%%%%%%%%%%%%%%%%%%%%%%%%%%%%%%%%%%%%%%%%%%%%%%%%%%%%%%%%%%%
\begin{abstract}
Implement hierarchical view frustum culling for large scale scenes consisting of triangles. First, construct a bounding volume hierarchy (BVH) using top-down method, middle point subdivision. Avoid rendering such BVH nodes that cannot be visible (out of viewing frustum) usually known as view frustum culling.
\end{abstract}

  % Klicova slova k uloze
\begin{keyword}
view frustum culling, bounding volume hierarchy, middle point subdivision
\end{keyword}

\end{frontmatter}

%\maketitle

%% \input{./section1.tex}
%% \input{./section2.tex}
%% \input{./section3.tex}
%% \input{./section4.tex}
%% \input{./section5.tex}

%%%% -------------------------------------------------------- 
%\section{\label{SEC:Intro}Úvod}
\section{\label{SEC:Intro}Introduction}

Tady bude krátký úvod do semestrální práce, motivace a cíle.

%%%% -------------------------------------------------------- 
%\section{\label{SEC:Description}Popis algoritmu}
\section{\label{SEC:Description}Algorithm Description}

Zkrácený slovní popis algoritmu, je možné použít i pseudokódy,
například balík {\textsc algorithm2e.sty}, pak je nutné odkomentovat
v~tomto souboru řádek:\\

{\textsc \textbackslash usepackage\{algorithm2e\}}.\\

Dokumentace k psaní pseudokódů na adrese:\\

\url{http://www.ctan.org/tex-archive/macros/latex/contrib/algorithm2e/algorithm2e.pdf}

Příklad pseudokódu algoritmu je uveden na obrázku~\ref{pseudo-code-rec}:

\input{rec.tex}

Minimální délka popisu algoritmu 3/4 stránky.

%%%% -------------------------------------------------------- 
%%\section{\label{SEC:Pitfalls}Potíže při implementaci}
\section{\label{SEC:Pitfalls}Implementation details}

Tady uvedete, co se Vám nedařilo a dařilo, s čím byly problémy a další
záležitosti týkající se implementace semestrální práce, či problém
čtení, konverze, či nalezení vhodných dat pro úlohu. Dále kde jste
strávili nejvíce času, kterou chybu v implementaci jste nejdéle
hledali a podobně. Časové nároky v hodinách (hodina = 60 minut) pro
vypracovaní semestrální práce včetně tohoto reportu. Rozdělení práce
mezi členy týmu, je-li úloha vypracována týmově.

%%%% -------------------------------------------------------- 
%\section{\label{SEC:Results}Naměřené výsledky}
\section{\label{SEC:Results}Results}

V této sekci budou tabulky, grafy, obrázky scén a text popisující
naměřené výsledky. Nedílnou součástí této sekce bude komentář
k~naměřeným výsledkům, například, jestli naměřená data odpovídají
teorii. Pro rychlost algoritmu je dobrá varianta uvádět čas zpracování
na jeden dotaz, průměrný počet traverzačních kroků na jeden dotaz,
průměrný počet incidenčních operací na jeden dotaz, dále pak dobu
stavby datové struktury a to v závislosti na vstupních datech (scéně)
a nastavení algoritmu (hloubka stromu, ukončující kritéria atd.). Je
nutné uvést charakterizaci scény, alespoň počet primitiv, objektů ve
scéně.

Dále je nezbytné uvést konfiguraci počítače, na kterém probíhalo
testování, tedy typ procesoru (případně počet procesorů), taktovací
frekvenci, počet jader použitých v implementaci, velikost vyrovnávací
paměti (cache) procesoru, operační systém použitý během testování
(32/64 bit), velikost hlavní paměti, typ a verzi kompilátoru,
přepínače kompilátoru použité pro překlad programu pro účely měření
výsledků a další.\\

Příklad formátování tabulky viz tabulka \ref{tab:tab1}.

\begin{table*}[t]\footnotesize
\begin{center}
\begin{tabular}{| c | c || r | r | r || r | r | r | r |}
\hline
          &               &   \multicolumn{3}{c||}{no acceleration}     &   \multicolumn{4}{c|}{n=12}  \\
\cline{3-9}
          &                 & \#trav. & \#isect. &      & \#trav. & \#isect. &  & \\
model & resolution &  steps    &    tests   & time  & steps    &    tests   & time   &  speedup \\
          &                  & $[per~ray]$ &  $ [per~ray]$   & [ms]  &$ [per~ray]$    &   $[per~ray]$   & [ms]  & [\%]  \\
\hline
\hline
Bunny & $128\times128\times100$ & 76.39 & 529 & 66.4 & 82.59 &  348 & 59.4 & 10.5 \\ \hline
Bunny & $256\times256\times200$ & 111.53 & 542 & 82.8 & 116.63 &  345 & 72.8 & 12.2 \\ \hline
Bunny & $512\times508\times400$ & 185.86 & 472 & 123.4 & 191.37 &  292 & 108.9 & 11.7 \\ \hline
Dragon & $128\times92\times60$ & 66.94 & 426 & 56.6 & 81.45 &  365 & 56.7 & -0.2 \\ \hline
Dragon & $256\times184\times116$ & 97.08 & 433 & 68.6 & 107.91 &  333 & 64.7 & 5.7 \\ \hline
Dragon &  $512\times364\times232$ & 15.81 & 377 & 96.5 & 166.93 &  264 & 87.0 & 9.9 \\ \hline
Buddha & $56\times128\times56$ & 7.34 & 104 & 16.6 & 10.15 &  100 & 17.9 & -7.9 \\ \hline
Buddha & $108\times256\times108$ & 11.10 & 116 & 19.7 & 14.71 &  107 & 21.2 & -7.6 \\ \hline
Buddha & $212\times512\times212$ & 15.81 & 96 & 24.6 & 19.63 &  84 & 26.0 & -5.9 \\ \hline
\end{tabular}
\end{center}
\vspace*{0mm}
\caption{{\label{tab:tab1}}Příklad tabulky. Další popis i upřesnění
  parametrů může následovat v této legendě.}
\vspace*{0mm}
\label{shadowtable}
\end{table*}

Příklad vložení souboru PDF je uveden na obrázku~\ref{fig:fig1} a ~\ref{fig:fig2}.

\begin{figure*}[!ht]
\begin{center}
  \includegraphics[width=0.3\textwidth]{cvut-logo-bw}
\caption{{\label{fig:fig1}}Příklad vloženého obrázku v PDF.}
\end{center}
\end{figure*}

\begin{figure*}[!ht]
\begin{center}
  \includegraphics[width=0.3\textwidth]{cvut-logo-bw}
  \includegraphics[width=0.3\textwidth]{cvut-logo-bw}
\caption{{\label{fig:fig2}}Příklad vložených obrázků v PDF vedle sebe.}
\end{center}
\end{figure*}

%%%% -------------------------------------------------------- 
%\section{\label{SEC:Conclusion}Závěr}
\section{\label{SEC:Conclusion}Conclusion}

V této části uvedete závěr, doporučení pro opakování implementace této
semestrální práce, omezení implementace či co se nepodařilo.

%%%% -------------------------------------------------------- 
%\section*{\label{SEC:ACK}Poděkování}
\section*{\label{SEC:ACK}Acknowledgment}

Práce je samostatná, ale pokud chcete někomu poděkovat například za
to, že Vás poslouchal při předběžné prezentaci či za přečtení a
kontrolu této zprávy a nalezení některých chyb, tak to tady uveďte.

%******************************************************************
% Tady nasleduje seznam literatury a to bud s pouzitim
% rucne usporadaneho seznamu ci s pouzitim system bibtex (prikaz
% \bibliographystyle{alpha} a \bibliography{dpgreport.bib}

%\bibliographystyle{abbrv}
%\bibliographystyle{alpha}
%\bibliography{dpgreport.bib}

% Druha moznost - rucne usporadany seznam literatury
\label{SEC:References}
%\section{Reference}
\renewcommand\bibname{References}
\begin{thebibliography}{10}

\bibitem[1]{Latex}
A. Balvínová a M. Bílý.
\newblock Textové informační systémy, sázecí systém \LaTeX, cvičení.
Skripta ČVUT-FEL, 1995.

\bibitem[2]{Danco}
V. Dančo.
\newblock Kapesní průvodce počítačovou typografií. Nakladatelství
Labyrint, Praha 1995.

\bibitem[3]{Rybicka}
J. Rybička.
\newblock \LaTeX\/ pro začátečníky. Nakladatelství Konvoj, 1999.

\bibitem[4]{GRUBER}
D. Gruber.
\newblock Kdo to má všechno číst, Nakladatelství Gruber, 1991.
(ISBN 80-900680-1-4).

\bibitem[5]{CSTUG} CSTUG: Úvod do \LaTeX{u}.
\newblock
\url{http://www.cstug.cz/latex/lm/frames.html}. Stránky z~roku 2009.

\bibitem[6]{CSTUGDOC}
CSTUG: Dokumentace a manuály k \LaTeX{u}.
\newblock \url{http://www.cstug.cz/documentation/index.html}

\end{thebibliography}

\end{document}
